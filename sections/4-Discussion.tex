\chapter{Discussion}

\section{Technical methods discussion}

The development of cancer is a sophisticated process. In the study of cancer mechanisms, driver mutations mentioned in section \ref{subsec:Variants type appeared in cancer tissues} are important for discovering oncogenic pathways. In this study, a new study flow was designed specifically for the existing sequencing data. Firstly, a direct mutation-calling workflow was applied to the tumor samples. Technically, all the variants that appeared in tumor samples and were different from the reference genome are able to be called. After a systematic variants quality control and filtration, the passed variants site and genotypes were tested with tumor sample's gene expression. The idea of this workflow is based on normal eQTL analysis that tests the association between normal sample genotypes and normal sample expression. This type of regular eQTL analysis is based on the absolute expression value change of the sample. For this reason, the absolute gene expression change should perform the association test with absolute genotype change. Usually, the genotypes of samples have a two-stage change. First, the normal samples always have some different genotypes with the reference genome. Then, the tumor samples always have different genotypes from normal samples. Consequently, the genotype variation in either normal-reference pair or tumor-reference pair is all absolute genotype changes. Conversely, the genotypes called between tumor and normal samples were relative genotypes change. This might cause a significant loss in the eQTL test. 

Based on the former illustration, the association test was performed between tumor sample genotypes and tumor gene expressions. The filtering steps used some common principles and thresholds that apply in population genomics, like minor allele frequencies and missingness, to ensure the quality of the test. Most variant sites were filtered because of >25\% genotype missingness. This is to keep the rest sites have enough confident genotype information between individuals for an eQTL analysis. Meanwhile, the tumor samples with matched normal samples provided the possibility to call somatic mutations separately. After the filtration, the distribution of somatic mutation load is similar to another research by Liu et al.\cite{liu_association_2018}. The somatic mutations were used to compare with the significant eGenes detected before and find the recurrently mutated eGenes.

\section{Results discussion}

\subsection{Mutation calling results}

The mutation calling result of tumor tissue using HaplotypeCaller implicated that most mutations are not harmful for gene function. A lot of them fall into the intron region or other non-protein-coding regions. There were above 99.6\% of mutations that did not change the type of coded amino acids. In comparison, the Mutect2 algorithms combined normal tissues and tumor tissues to find somatic mutations, and the proportion of putative 'HIGH' impact mutations was higher than HaplotypeCaller's result. Meanwhile, the proportion of missense mutations was much higher than nonsense mutations, which suggests that mutations that happen during cancer cell development have higher possibilities to change the gene-coded amino acid type.

Previous known key ccRCC gene VHL (von Hippel Lindau gene) have a sum of 54 times mutation in the Mutect2 somatic calling results. The mutated VHL gene that lost the functions will cause a high level of HIFα factor. The HIFα factor controls multiple downstream gene expression, such as VEGF, PDGF and TGFα\cite{clark_role_2009}. The VEGF and PDGF family genes have also appeared in somatic mutation results. In the study by yang et al.\cite{yang_gene_2017}, the VHL gene has a significant association with differential expressed gene modules. In this study, the most significant eQTL site for VHL gene was a deletion (ATCT > A) with rs id rs5030648, which has a q value of 0.569 but did not achieve a significant q value below 0.05. Therefore, the VHL gene might has a hidden comprehensive disease-causing mechanism that different with the direct mRNA expression decrease. 

\subsection{The possible mechanism of identified significant eGenes results}

The identified significant eGenes have more than 35\% overlap with significant eGenes in GTEx kidney V8 release. These eGenes were significantly being controlled in two different cohorts of kidney tissue, one of which contained healthy tissue and the other contained cancer tissue. Often, it will be acceptable to have a relatively small overlap between two results, even if they all originate from kidney tissue. The population composition of these two cohorts is different, so the two cohorts might originally carry their own feature SNPs. Furthermore, even this study has mostly tried to replicate the same workflows of the GTEx database, but the different sequencing methods, the minor changes in details in workflows and different sample size could all affect the final eQTL results. Considering these variable factors, the overlapped eGenes still account for more than 35\% of total eGenes in this study. This high proportion of overlap suggests a high similarity in gene expression regulation between different kidney tissues. This kind of similarity might because of the same potential germline variants in both cohorts that controls the corresponding gene expression. Even tumor cells have changed a lot in gene expression, tumor cells will keep some extent of similarity with normal cells, especially for key genes such as housekeeping genes. These genes expression might controlled by specific genetic loci, that both identified in tumor cells and normal cells. 

On the other hand, significant gene regulation changes also appeared in cancer patients of the Japanese cohort. This change might be caused partially by population differences, but more likely caused by tumor-specific variation. 

The significant eGenes have shown a certain extent of relationships with ccRCC disease mechanisms. Among the four most significant eGenes, some showed relationships with ccRCC based on previous publications, but others did not. The gene LINC01291 is a long intergenic non-protein coding RNA gene, which has been reported with an impact that promotes the aggressive properties of melanoma. The boxplot(a) demonstrates the increase in expression of LINC01291 when the mutation occurred. It competes with specific miRNAs and increases the expression level of IGF-1R (insulin-like growth factor-1 receptor)\cite{wu_long_2021}. The insulin-like growth factor (IGF-1) pathway is also an important pathway in the development of ccRCC. The higher expression of IGF-1R leads to a higher risk of death for ccRCC patients\cite{tracz_insulin-like_2016}. The lncRNAs have been emphasized in the Introduction section for their important regulatory functions. Except LINC01291, another lncRNA named TRIM52-AS1 was discovered to function as a tumor suppressor gene for ccRCC according to the research by Liu et al.\cite{liu_downregulation_2016}. The TRIM52-AS1 is down-regulated in renal cell carcinoma (RCC) tissues, which suggests the low expression value of TRIM52-AS1 might lose its function as a tumor suppressor. TRIM52-AS1 was also identified as one of the eGenes in the Japanese cohort. The most significant eQTL site is chr5\_181260427\_CTCT\_C with dbSNP id rs200454506. Based on the boxplot drawn by the provided script, the happened deletion will cause a decrease in expression value, which might lead to tumor development. Notably, the LncRNA gene family has been identified not only as significant eGenes but also as highest recurrently mutated sites(BAGE2), which indicates the importance of lncRNA in gene expression regulation of ccRCC.

The second identified significant eGene ranked by q-value is ERAP2 (endoplasmic reticulum aminopeptidase 2). ERAP2 encodes multifunctional enzymes that have a great biological function when cell-generating major histocompatibility complex (MHC) class I binding peptides\cite{compagnone_regulation_2019}. This process might influence tumor immunogenicity. Research about anti-PD1 response in ccRCC \cite{au_determinants_2021}have identified ERAP2 mutation as a signature gene related to defective antigen presentation. The boxplot(b) in Figure \ref{boxplots} showed that the mutation of ERAP2 decreased the gene expression level so that potentially decreased the level of enzymes. 

Some significant eGenes have not been reported in ccRCC research before. The gene RPS26 (Ribosomal protein S26) is a disease-related gene, and the mutation of RPS26 usually causes Diamond-Blackfan Anemia 10 (DBA10)\cite{doherty_ribosomal_2010}. This gene functions in pre-rRNA processing but no evidence showed its relation with kidney cancer in previous studies. Notably, the RPS26 protein has been implicated in regulating the p53 response to DNA damage\cite{cui_ribosomal_2014}. The p53 protein mostly has great impacts on different cancers, which suggests RPS26 as a newly discovered key gene for ccRCC. The gene XRRA1 (X-Ray Radiation Resistance Associated 1) is also a gene related to DNA damage repair. A few works of literature mentioned that it might be related to cancer development, such as the study by Wang et al.(2017)\cite{wang_xrra1_2017}, that showed XRRA1 targets ATM/CHK1/2-Mediated DNA Repair in Colorectal Cancer.

\subsection{The possible mechanism of identified 13 recurrently mutated eGenes}

All the 13 eGenes are somatically mutated at a frequency of at least five times in all the 100 patients. The Ubiquitin Specific Peptidase 24 (USP24) and Ubiquitin-specific protease 6 (USP6) are belongs to the same family of ubiquitin-specific peptidases (USPs). USPs play a significant role in a wide variety of diseases, and a growing number of studies reveal that USPs are crucial for cancer progression. Specifically, USP24 blocks the progression of the cell cycle from metaphase to anaphase, resulting in cell cycle arrest. It is a novel tumor suppressor identified by Bedekovics et al.(2021)\cite{bedekovics_usp24_2021}. USP6 is involved in activating the NFκB pathway, thereby positively influencing tumorigenesis\cite{young_role_2019}. Additionaly, USP6 is a key mutated gene in the VHL-negative ccRCC patients\cite{tan_establishment_2013}. Thus, the corresponding eQTLs for these two eGenes that were identified in this study have high potential to play the role as driver mutations in ccRCC.

The Leucine Rich Repeats And IQ Motif Containing 3 (LRRIQ3) is a protein coding gene. It is down-regulated in a number of cancers based on the tissue expression data from The Human Protein Atlas. Laminin Subunit Gamma 2 (LAMC2) encodes a part of protein called laminin 5. Laminins are a group of proteins that regulate cell growth, cell motility and cell adhesion. The silencing of LAMC2 will inhibit angiogenesis, which is a crutial driver in ccRCC pathogenesis\cite{pei_silencing_2019}\cite{seles_long_2020}. The mutation of ALMS1 Centrosome And Basal Body Associated Protein (ALMS1) causes Alstrom syndrome which is a rare genetic disorder, but the encoded protein has multiple functions such as maintaining the cohesion and composition of centrosomes, organizing the actin cytoskeleton\cite{hearn_alms1_2019} and regulating the TGF-β signaling pathway\cite{alvarez-satta_alms1_2021}. The Nucleoporin 205 (NUP205) have been identified to be drivers of lung cancer\cite{fujitomo_critical_2012}. Tripeptidyl-peptidase 2 (TPP2) has been discovered to be significantly up-regulated in oral squamous cell carcinoma derived cells. Also, the overexpression of TPP2 leads to accelerated cell growth and resistance to apoptosis\cite{tomkinson_tripeptidyl-peptidase_2019}. The Neuroblastoma Breakpoint Family Member 9 (NBPF9), Fc receptor-like protein 3 (FCRL3), ARFGEF Family Member 3 (ARFGEF3), Dynein Axonemal Heavy Chain 11 (DNAH11)and Rabphilin 3A (RPH3A) are the genes that has not been reported as cancer related genes before. PMS2P1 is a pseudogene so we didn't discuss it here. 

\subsection{The discussion about pathways enriched by Enrichr}
The Enrichr results have given two important potential disease-related pathways: ECM-receptor interaction and Homologous recombination. ECM-receptor interaction pathway is a cancer-related pathway. The role of ECM has been proved in different types of cancers, such as prostate cancer, colorectal cancer, and breast cancer\cite{bao_transcriptome_2019}. Homologous recombination(HR) has been proved to be highly related to ccRCC. SETD2, one of recurrently mutated genes, has been proved to affect the efficiency of homologous recombination repair in ccRCC through the loss-of-function mutations\cite{kanu_setd2_2015}. 

\subsection{Conclusion}

As a conclusion, this study has provided a database of newly identified significant eGenes and eQTLs of the Japanese ccRCC patient cohort. The database provides a reference for future analyses. Several potential key genes have also been identified. The LINC01291, for instance, is a gene with a high significance that is related to the development of ccRCC. We also derived a list of recurrently mutated significant eGenes based on the results of somatic mutation calling and eQTL analysis. Some of these have just been discovered, and others have already been proven to be closely related to ccRCC and could be potential drug targets, such as USP24 and USP6. It is more likely these 13 eGenes will serve as driver genes due to the presence of somatic mutations.  In summary, this study investigated tumor-specific eGenes and eQTL sites in comparison to healthy kidney eGenes, and achieved a combined analysis through the detection of somatic mutations within tumor cells to identify recurrently mutated eGenes as potential driver genes for ccRCC.

\section{Delimitations}

This project has used the whole-exome sequencing data as genotype data, so only variants located inside genes or near genes were able to be detected. Since this project focused on cis-eQTL mapping, the whole-exome sequencing should not cause a large impact in principle. In the variants filtering step, several samples have a relatively high missing genotypes rate based on Plink and SnpEff, but were not excluded in the downstream analysis.  In this way, a complete cohort of 100 patients was kept. But some missing genotypes inside those individuals have been kept in the variant sites. After generating the tumor-specific eQTLs table, the downstream analysis changed to focus on gene-level (eGenes)rather than mutation level. As a result, future work is needed to complete the information on eQTL level.