\newpage
\thispagestyle{plain}
~\\
\vfill
{ \setstretch{1.1}
	\subsection*{Authors}
	Author Name: Xiya Song\\
	Molecular Techniques In Life Science\\
	KTH Royal Institute of Technology
	
	\subsection*{Place for Project}
	Stockholm, Sweden\\
	
	\subsection*{Supervisor }
	The Supervisor: Dr.Cheng Zhang\\
	Principal investigator: Professor Adil Mardinoglu \\
	Place: Stockholm, Sweden \\
	KTH Royal Institute of Technology
	~
}


\newpage
\thispagestyle{plain}
%%%%%%%%%%%%%%%%%%%%%%%%%%%%%%%%%%%%
%%  The English abstract          %%
%%%%%%%%%%%%%%%%%%%%%%%%%%%%%%%%%%%%
\chapter*{Abstract}
%%%%%%%%%%%%%%%%%%%%%%%%%%%%%%%%%%%%

Clear cell renal cell carcinoma (ccRCC) is the most common form of kidney cancer in adults. Further investigation of the molecular mechanisms and underlying processes of ccRCC is crucial for targeted drug development. Expression quantitative trait loci (eQTL) analysis is a method to identify genetic variants that might affect gene expression. The identified eQTLs and genes being regulated (eGenes) from tumor tissue provide valuable information concerning cancer development. In this study, the DNA-sequencing and RNA-sequencing data of 100 Japanese ccRCC patients were used to perform an eQTL analysis using the methodology developed by the Genotype-Tissue Expression (GTEx) project. A list of 805 significant eGenes with corresponding most significant eQTLs has been identified for the Japanese ccRCC cohort. Compared to the healthy kidney eQTLs database in the GTEx database, there are 518 new eGenes that were only discovered in the Japanese cancer cohort. Furthermore, 13 new eGenes are associated with recurrently mutated somatic sites, indicating that they may be driver genes in cancer development. The long non-coding RNA (lncRNA) genes and the Ubiquitin-specific protease (USPs) family have shown an important role in ccRCC based on the results. In summary, the study provides a comprehensive database of the eGenes and eQTLs specific to Japanese ccRCC patients and identified some key genes as potential drug targets.

\subsection*{Keywords}
eQTLs, eGenes, clear cell renal carcinoma(ccRCC),GTEx, kidney cancer, somatic mutations

\newpage
\thispagestyle{plain}
\chapter*{Acknowledgements}

I would like to express my gratitude to all those who helped me during the writing of this thesis. First of all, I want to thank my supervisor, Dr.Cheng Zhang, for his continuous encouragement and guidance during the period of this project. Also, he has provided me the ground idea of the whole research and all the resources needed in this degree project. I am also greatly indebted to professor Adil Mardinoglu who gave me the opportunity to enter the sysmedicine lab. I am also very grateful to all the Ph.D students and postdocs researcher in the lab, who gave me kindly advice and warming help, especially for Dr.Xiangyu Li who helped me with handling specific details during the project, and Ph.D student Hong Yang who gave me the help on software installation and other tough obstacles. I also want to thank all the teachers and my classmates in this two-year's master study who gave me a great study experience. Last my thanks would go to my beloved family for their loving considerations and great confidence in me all through these years.


\newpage

\etocdepthtag.toc{mtchapter}
\etocsettagdepth{mtchapter}{subsection}
\etocsettagdepth{mtappendix}{none}
\thispagestyle{plain}
\tableofcontents

\newpage


